\section{Conclusion}

In this report, we have presented our Flyback Converter project design details for the Hardware Project of EE464-Static Power Conversion II course. We started this project aiming that we would have implement this project on a hardware level, and gain a lot of experience while doing it. This was actually the main purpose of this hardware project. We are hoped to experience the learn by doing philosophy of the power electronics area. However, due to the unfortunate Covid-19 pandemic situation, we could not complete and implement our project on hardware level. For that reason, we are asked to finish our project by doing our detailed designs and simulations on computer based environments. This new online education period taught all of us to stand up against new challenges, and set sails for new horizons by adapting our style, approach and motivations.

Therefore, this report presented the best of our team's work on completing this project under the current circumstances. In this report, we have expressed the fundamentals of our design, parameter calculations, component selections and the simulation results of our project. In the first part of the report, we have explained the design methodology of the Flyback Converter topology. We have also stated our design goals for this project. Transformer parameter verification is also presented in the first part of the report. In the second part, we have showed the magnetic design steps for the Flyback Converter. We have made our magnetic design by selecting an appropriate core and cable for the transformer. In the third part, we have presented the simulation results of the constructed Flyback Converter topology including the non-idealities of the Flyback Converter components and the designed transformer. The simulation results for two different source voltage level have been presented. In the fourth part, the component selections for the converter parameters have been expressed. The component selection for the analog controller for the closed loop control and the semiconductor devices and filter components have been done in this part. In the selection of the MOSFET, diode and the output filter capacitor, we have paid attention to the voltage and current stresses over these devices observed during the simulations of the circuit. In the fifth part, we have presented the analytical calculations and the simulation results for the efficiency of the constructed Flyback Converter circuit. The analytical efficiency value of the converter has been computed for three different operating condition and compared with the respective efficiency result obtained from the simulations of the converter circuit in Simulink. In the sixth part, the thermal analysis has been shown. The thermal analysis has been conducted for the semiconductor devices: MOSFET and diode considering their worst case operating conditions. Then, according to the results of the conducted thermal analysis, we have selected proper heat sinks for the MOSFET and the diode. In the seventh part, the design steps for the closed loop control system design for the non-ideal Flyback Converter have been explained. First of all, the AC equivalent small signal modeling of the Flyback Converter has been shown by following the derivations. Then, the transfer function derivations by using the obtained small signal AC equivalent circuit model have been explained in detail. Next, the derived transfer function has been modified to include the non-idealities, and design a compensator for the closed loop control of the Flyback Converter. Finally, the designed compensator has been simulated in LTSpice simulation environment in order to verify the desired operation of the designed closed loop converter system. The simulation results for different load change conditions and step input change conditions have been presented at the end of this part. The obtained results have been discussed and analyzed regarding whether they comply with the analytical calculations, our design objectives and the project requirements. In the final part of the report, the PCB design steps for the constructed closed loop Flyback Converter circuit has been expressed. The PCB design details, tips and guidelines have been explained in this part. The KiCad schematic of the constructed closed loop Flyback Converter has been shown. The PCB layout of the designed converter circuit has also been shown. The 2D and 3D views of the designed PCB model of the circuit have been presented in this part of the report. Finally, we have also included the Bill of Materials (BOM) for the components we have used, and the manufacturing quote taken from PCBWay for manufacturing 1000 pieces of our designed Flyback Converter in this part of the report. 

In summary, this project maybe could not help us to gain practical experiences on constructing, implementing and experimenting on a practical DC-DC converter design, but it gave us the opportunity to improve our simulation and PCB design skills. We have learned how to make detailed simulations, and how to utilize various simulation software programs for conducting these simulations. We have also gained huge experience on designing our own analog controller (compensator) for the closed loop control of the DC-DC converters. We had the chance to have an idea about the PCB design work, and gain a lot of experience on how to make a PCB design for a converter circuit. We have learnt the details, tips and the guidelines of the PCB layout design. We had the chance to become familiar with the PCB design tools like KiCad and Altium. We have experienced how to best utilize from various simulation tools and softwares for our project design and verification.