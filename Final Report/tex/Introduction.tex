\section{Introduction}

One of the most commonly used circuits to perform DC-DC conversion in power electronics applications are Switching Mode Power Supplies (SMPS). To provide a safer output and to comply with international standards of isolation of the output side from the input side is required, especially in grid-connected applications. There are two main topologies of the dc-dc converters with galvanic isolation, which are the Flyback Converters and the Forward Converters. These topologies provide galvanic isolation between their input and output thanks to the isolation transformer placed among the two sides. The isolation transformer provides electrical isolation between the input and output sides of the converter. The isolation transformer also provides larger bandwidth or operation range on the voltage transfer capabilities of the converter by adjusting the primary to secondary turns ratio. Hence, these topologies reduce the weight of the duty ratio on the voltage transfer capability by having the turn ratio of the transformer to make the most of the work. The Flyback Converter topology requires less number of components compared to the Forward Converter. Hence, it has the advantage of reduced cost and size, and higher power density. 

In the previous report, we have presented our topology selection, and the reasoning behind selecting that topology together with the simulation results of the selected topology. We have stated our topology selection as FLY\#1.

Due to the Covid-19 pandemic situation, we, Şönt A.Ş., unfortunately did not have the possibility to complete and implement this project on hardware level. Therefore, in this Final Report, we will provide our project results based on computed aided design and simulations.

In this report, fundamental points of our design, parameter calculations, simulation results with the calculated parameters, component selections, closed loop control system design and key components for building the Flyback Converter topology are explained. First of all, the design methodology of the Flyback Converter is given with the design goals and parameter calculations. The verification of the transformer parameters is also presented in the first part. The magnetic design of the converter transformer is explained in detail in the following section of the report. The transformer core selection and the cable selection for the transformer windings are presented in this section. Next, the simulation results of the constructed Flyback Converter topology depending on the design selections and computed design parameters are given in the Simulation Results section of the report for different source voltage values. Following the simulation results, the component selections for the analog controllers, the semiconductor devices and filter elements in the Flyback Converter topology are presented according to the calculated design parameters and the obtained simulation results. The commercial product selections for the converter switch, diode and output filter capacitor according to the maximum voltage and current stresses over them are given in this section of the report. In the next section of the report, the Efficiency Analysis of the constructed Flyback Converter is presented. The efficiency is analytically calculated for different operating points with different source voltage values, and then compared to the corresponding efficiency values obtained from the simulations of the converter circuit in Simulink for the same operating points. Then, the Thermal Analysis for the converter circuit components is given. The heat sink selections for the MOSFET switch and the diode of the Flyback Converter are shown in this section according to the conducted thermal analysis. In the following section of the report, the closed loop controller design steps for the Flyback Converter circuit are presented. First of all, the derivations for obtaining the AC equivalent circuit model of the Flyback Converter are expressed. The AC equivalent circuit model of the Flyback Converter is presented in the first subsection. Then, the transfer function derivation steps using the obtained AC equivalent circuit model are shown in the following subsection. Following the derivations of the transfer function of the converter topology, the compensator design operation by using the derived transfer function and by considering the possible converter circuit component non-idealities is shown. Finally, the designed closed loop compensated Flyback Converter is simulated and tested under different load change and step input change conditions. The obtained simulation results are presented in the final subsection of this section of the report. In the next section of the report, the steps of the PCB design for the constructed closed loop Flyback Converter circuit are presented. The PCB design details, tips and guidelines are expressed, and the obtained 2D and 3D schematics are shown with the constructed PCB layout of the converter circuit. Finally, the report is concluded by giving the acknowledgements and the final concluding remarks of the project.

