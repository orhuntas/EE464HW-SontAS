\section{Topology Selection}

\textbf{Selected Topology: FLY\#1}

We, as Şönt A.Ş., selected the Flyback Converter topology FLY\#1 over Flyback Converter options and Forward Converter topology for a number of reasons. First of all, it includes less number of components compared to the Forward Converter topology. Hence, the cost is reduced compared to the Forward Converter topology. Another reason is that it does not require an output filter inductor, which makes the magnetic design phase easier compared to the Forward Converter topology. As a result, there is no need for a second core selection and extra magnetic design for an output filter inductor. In terms of these aspects, the Flyback Converter topology is the simplest topology among the other options. \par Furthermore, the voltage gain control of the Flyback Converter topology in DCM operation is easier than that in the Forward Converter topology. As a result, the current mode control and the voltage mode control of the Flyback Converters can be made in DCM operation, which also helps to reduce the current stress over the MOSFET and the diode during switching since the switching is done when the current is equal to zero. Zero current switching also eliminates the reverse recovery losses over the output rectifier diode. Furthermore, the DCM operation allows switching the MOSFET when the voltage across the drain to source terminals of the MOSFET is approximately equal to the input voltage. This greatly reduces the switching losses of the converter since the switching losses are proportional to the square of the drain to source voltage of the MOSFET. \par The availability of DCM operation also enables the valley switching operation, which further decreases the switching losses by making it possible to switch the MOSFET while the drain to source voltage is at its minimum below the input voltage. This operation requires frequency modulation (FM) controlled by a controller. Since we found the UCC28740 to be widely available, inexpensive and capable of completing the tasks given in the project, we decided to build our Flyback Converter around it. The EMI effects are also reduced with the valley switching operation.  All in all, these operation modes make the Flyback Converter topology obtain a higher efficiency than Forward Converter topology. One another advantage of the availability of DCM operation is that it helps to reduce the required inductance value for the transformer, which helps to reduce the transformer size and weight. If we evaluate the two topologies in terms of voltage stress over the switching MOSFET, the voltage stress over the switching MOSFET in Flyback Converter topology is less than that in the Forward Converter topology. These points and considerations conclude our selection of Flyback Converter topology over Forward Converter topology.

Now, the final topology selection among the three possible Flyback Converter topology options FLY\#1, FLY\#2 and FLY\#3 will be explained.

If we compare the Flyback Converter topology options FLY\#1, FLY\#2 and FLY\#3 in itself, we did not want to work with AC input voltage since it would require an extra input full bridge rectifier together with an very large filter capacitor at the rectifier output. This eliminated the option FLY\#3 for us. Then, between FLY\#1 and FLY\#2, we selected the FLY\#1 option since its output voltage is higher than that of FLY\#2 option. This makes the average output current in FLY\#1 less than that in the FLY\#2, which helps to increase the efficiency in the FLY\#1 option compared to FLY\#2 option. Also, the current ripples and maximum current ratings will be smaller in FLY\#1 due to smaller average output current. In addition, the higher output voltage for the same output voltage ripple limit of 4\% helps to choose the output filter capacitor smaller in FLY\#1 option, which reduces the filter size. For these reasons, we selected the FLY\#1 option.