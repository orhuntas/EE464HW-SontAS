\section{Introduction}

One of the most commonly used circuit to perform DC-DC conversions are Switching Mode Power Supplies (SMPS). To provide a safer output and to comply with international standards isolation of output from the input side is required, especially in grid-connected applications. There are two main topologies of the dc-dc converters with galvanic isolation which means there is no electrical contact between two sides are the Flyback Converters and Forward Converters. These topologies provide isolation thanks to the isolation transformer placed between the two sides. The isolation transformer also provides larger bandwidth or operation range on the voltage transfer capabilities by adjusting the primary to secondary turns ratio. Hence, these topologies reduce the weight of the duty ratio on the voltage transfer capability by having the turn ratio of the transformer to make the most of the work. 

In this report, fundamental points of our design, parameter calculations. simulation results and key components for building the Flyback Converter topology are explained. First of all, the reasoning behind the topology selection is presented. Then, the design methodology of the Flyback Converter is given with the design goals and parameter calculations in the following part. The magnetic design of the transformer is also be explained in detail. Depending on the design selections and computed parameters, the simulation results of the constructed topology is presented in the Simulation Results part of the report. Finally, the component selections for the semiconductor devices are included according to the maximum current and voltage stresses over them, which are computed by the analytical calculations and observed during the simulations. The magnetic core selection for the transformer is can be found in that part of the report too. Furthermore, the appropriate cable selection both for the primary and secondary sides of the transformer is done considering the effects like the skin effect and the proximity effect together with the maximum currents flowing through the transformer windings.
